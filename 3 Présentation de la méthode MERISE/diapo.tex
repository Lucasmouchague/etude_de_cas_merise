\documentclass{beamer}
\usepackage[utf8]{inputenc}
\usetheme{Warsaw}

\title{Présentation de la méthode Merise}
\institute{Ynov Bordeaux}

\begin{document}

\begin{frame}
\titlepage
\end{frame}


\begin{frame}
\frametitle{Historique}
Méthode française.

Opérationnel depuis les années 80.

Résultat des travaux de Hubert Tardieu.
\end{frame}

\begin{frame}
\frametitle{Approche de la démarche}
3 axes:
\begin{itemize}
    \item Cycle de vie.
    \item Cycle de décision.
    \item Cycle d'abstraction.
\end{itemize}
\end{frame}
\begin{frame}
\frametitle{Cycle de vie}

Maitrise de la chronologie des opérations.

Série de comptes-rendus:

\begin{itemize}
    \item Schéma directeur.
    \item étude préalable.
    \item étude détaillée.
    \item étude technique.
\end{itemize}
\end{frame}

\begin{frame}
\frametitle{Cycle de décision}
Jalons de validation.

Importance d'un calendrier entre les differents acteurs.
\end{frame}

\begin{frame}
\frametitle{Cycle d'abstraction}
Dissociation des communications, des données et des traitements.

Trois niveaux de questionnement:
\begin{itemize}
    \item Quoi? M.C.D et M.C.T
    \item Qui, quand, oû? M.L.D, M.O.T
    \item Comment? M.P.D
\end{itemize}

\end{frame}
\begin{frame}
\title{Avez vous des questions?}
\titlepage
\end{frame}
\end{document}